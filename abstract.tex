The Timed Multiagent Patrolling is a complex multi-agent task, which requires 
a group of agents to coordinate each other’s actions in order to obtain optimal 
results for the whole group. Patrolling a country’s borders, watching the 
corridors of a building, monitoring maritime fleets, inspecting areas that may be 
subject to gas leakage or fires are examples of such a patrolling task. An 
efficient solution to the Multiagent Patrolling can contribute in a variety of 
domains such as computer network management, web search engines and traffic 
inspection.

Previous works have proposed many solutions to patrolling efficiently with a group 
of agents. Heuristic agents, game theory based agents, negotiation mechanisms, 
reinforced learning techniques, gravitational strategies, 
ant colony based agents and hybrid evolutionary and ant colony optimization approaches have all been applied to 
solve the multi agent patrolling problem. This work aims to contribute to the study 
of the patrolling task by developing pure evolutionary approaches and comparing them 
to other strategies proposed in the literature through simulations.

The empirical evaluation of the strategies will be made using benchmarks proposed by 
other researchers in previous publications, software freely available on the 
internet and maintained by the community of researchers, and the patrolling simulator 
named Simple Patrol provided by researchers from the 
Universidade Federal Rural de Pernambuco.

\begin{keywords}
autonomous agents, multi agent systems, coordination and patrolling, evolutionary 
algorithms
\end{keywords}