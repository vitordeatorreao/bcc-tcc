A Patrulha Multiagente Temporal é uma complexa tarefa multiagente, a qual 
requer que um grupo de agentes coordene as ações uns dos outros a fim de 
obter os melhores resultados para todo o grupo. Alguns exemplos de patrulha 
são: patrulhar as fronteiras de um país, vigiar os corredores de um prédio, 
monitorar frotas marítimas e inspecionar áreas que possam ser sujeitas a 
vazamento de gás ou incêndio. Uma solução eficiente para a Patrulha 
Multiagente pode contribuir em uma variedade de domínios, tais quais, 
administração de redes de computadores, motores de busca da Web e 
fiscalização do tráfego.

Trabalhos recentes propuseram diversas soluções para a patrulha eficiente 
feita por um grupo de agentes. Agentes heurísticos, agentes baseados em 
teoria dos jogos, mecanismos de negociação, 
técnicas de aprendizado com reforço, estratégias gravitacionais, agentes 
baseados em colônia de formigas e abordagens híbridas evolucionárias com \textit{Ant Colony Optimization} já foram 
aplicados para solucionar o problema da patrulha multiagente. O presente 
trabalho visa contribuir para o estudo do problema da patrulha através do 
desenvolvimento de estratégias puramente evolucionárias e sua comparação, 
através de simulação, com outras estratégias presentes na literatura.

A avaliação empírica das estratégias será realizada utilizando benchmarks 
propostos por outros pesquisadores em trabalhos anteriores, software 
disponível de graça na internet e mantido pela comunidade de pesquisadores da 
área, e o simulador chamado \textit{Simple Patrol} disponibilizado pelos pesquisadores 
da Universidade Federal Rural de Pernambuco.

\begin{keywords}
agentes autônomos, sistemas multiagentes, coordenação e patrulha, algoritmos 
evolucionários
\end{keywords}