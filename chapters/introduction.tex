\chapter{Introduction}
\label{chp:introduction}

% \begin{quotation}[]{Poul Anderson}
% I have yet to see any problem, however complicated, which, when looked at in the
% right way, did not become still more complicated.
% \end{quotation}

Many activities related to vigilance, inspection and control, which are currently 
accomplished by humans, will possibly be executed by autonomous systems in the future 
\citep{hernandez2013game}. Some examples of such tasks are patrolling a country’s borders 
or walls of a civil area, watching a building’s corridors, monitoring maritime fleets and 
inspecting areas subject to gas leakage or fires \citep{sampaiophd}.

According to \citep{hernandez2013game}, these security systems, when operated by human 
beings, are, in their majority, predictable and inflexible, because the operator’s performance 
can be influenced by factors such as fatigue, boredom and distraction. Therefore, the authors 
state that it is important to improve the elements of security in such systems to help the 
human operators. A multiagent patrol system is a prominent candidate to play that role 
\citep{Chevaleyre:2004:TAM:1018411.1019013}.

The multiagent patrol can be defined as the task where a group of agents must perceive a 
limited portion of the environment and detect events or anomalies \citep{6315145}. More 
specifically, it is defined as the problem where a team of individuals (agents) visits as 
frequently as possible points of interest contained in an area \citep{6495145}. Or it can 
also be defined as a vigilance problem, where the agents must minimize the time between 
visits on important locations inside a known environment \citep{Pippin:2013:PBT:2480362.2480378}.