\chapter{Trabalhos Relacionados}

\section{Considerações Iniciais}

Um dos grandes obstáculos para o estudo do Problema da Patrulha Multiagente é a 
falta de concordância, dentre os pesquisadores, sobre nomenclatura, escopo e 
critérios de avaliação para as soluções propostas. Isso pode se dever ao fato do 
problema da patrulha estar presente em diferentes áreas como Inteligência 
Artificial, Sistemas Multiagentes e até Robótica \citep{sampaiophd}.

Há, na literatura, nomes diferentes para o problema da Patrulha Multiagente. Por 
exemplo, alguns autores \citep{hernandez2013game} chamam o problema de “patrulha 
multirobô” (ou Multi-Robot Patrolling, em inglês), outros \citep{6495145} se 
referem ao problema pelo nome de “patrulha temporal” (tradução de timed 
patrolling), já alguns pesquisadores \citep{Koenig:2001:TCA:375735.376463} 
utilizam o termo “cobertura de terreno” (terrain coverage, em inglês). No 
entanto, no decorrer do presente trabalho será utilizado o termo “Patrulha 
Multiagente”, pois ele é considerado mais apropriado por diversos pesquisadores 
\citep{6900280}, \citep{sampaiophd}, \citep{hernandez2013game} e \citep{6315145}.

O escopo é outro fator que varia entre os trabalhos. \citep{6615158}, por 
exemplo, faz uma análise do problema da patrulha onde o ambiente é dinâmico. 
Isto é, o grafo onde os agentes patrulham sofre alterações ao longo da execução 
do agente. Em um dado momento, o grafo pode ter vértices adicionados ou 
removidos. Já \citep{6900280} trabalham levando em conta que agentes podem ter 
velocidades diferentes. Os autores de \citep{Pippin:2013:PBT:2480362.2480378} 
levam em consideração que agentes podem agir com eficiência abaixo do esperado, 
isto é, o agente não pode ser confiado para realizar a tarefa que lhe foi 
passada sem falhas. Em \citep{4209122}, os autores consideram restrições de 
frequência, isso significa que cada nó tem a si designado um valor de frequência 
com a qual o nó \textbf{deve} ser visitado. Outro exemplo seria o trabalho feito 
em \citep{6495145} e \citep{Poulet:2012:b}, onde é feita uma análise do problema 
em uma configuração de sistema aberto. Sistemas abertos foram definidos em 
\citep{6040660} como aqueles agentes podem entrar e sair a qualquer momento da 
execução.

Na presente pesquisa, será utilizado o escopo para a Patrulha Multiagente 
doravante referido como "padrão", onde os agentes possuem eficiência idêntica, 
são confiáveis e não são retirados nem adicionados ao longo da patrulha. Quanto 
ao ambiente, o objeto de estudo deste trabalho compreende apenas ambientes que 
permanecem estáticos durante a execução dos agentes, e os pontos que devem ser 
patrulhados não possuem restrições de frequência.

Existem diversas métricas que podem ser utilizadas para medir a eficiência de 
cada solução para a \ac{tmap}. Diferentes pesquisadores utilizam métricas 
distintas. Em \citep{Machado:2002:MPE:1765317.1765332} foram propostas as 
métricas mais utilizadas para a \ac{tmap}. São elas: ociosidade instantânea do 
nó, ociosidade instantânea do grafo, ociosidade máxima e tempo de exploração. 
Depois deste trabalho, \citep{sampaiophd} propôs uma nova família de métricas 
baseadas nos intervalos entre visitas. As métricas utilizadas nos trabalhos mais 
recentes variam: \citep{6900280} e \citep{Pippin:2013:PBT:2480362.2480378} 
utilizam o intervalo máximo, já \citep{4209122} fazem uso da frequência mínima e 
\citep{hernandez2013game} compara as frequências mínimas.

Dessa forma, esta sessão visa, através de pesquisa bibliográfica, explorar os 
trabalhos relacionados para construir a base de conhecimento necessária para 
formular uma proposta compatível com o objetivo deste projeto. Serão discutidas 
as soluções para o problema da patrulha presentes na literatura 
classificando-as por seu escopo e métricas utilizadas na avaliação dos agentes.

\section{Levantamento Bibliográfico}

\subsection{Classificações para a \ac{tmap}}

São notórios alguns trabalhos muito citados na literatura que visaram 
classificar as soluções para a \ac{tmap}.

Em \citep{Chevaleyre:2004:TAM:1018411.1019013}, as abordagens propostas até 
então foram classificadas em \textbf{cíclicas} (ou de ciclo único) e \textbf{
baseadas em particionamento}. As soluções de ciclo único são aquelas onde é 
calculado um ciclo que cobre todos os vértices do grafo, e então, os agentes são 
colocados para caminhar nesse ciclo indefinidamente. As abordagens baseadas em 
particionamento são aquelas onde o território a ser patrulhado é dividido em 
$r$ regiões, onde $r$ é o número de agentes. Os agentes devem, então, patrulhar 
dentro de suas respectivas regiões.

Já \citep{Machado:2002:MPE:1765317.1765332} fazem uma extensa classificação das 
arquiteturas de soluções para a \ac{tmap}. Eles dividem os agentes em: \textbf{
reativo} ou \textbf{cognitivo}; com comunicação \textbf{permitida} ou \textbf{
proibida}; com coordenação \textbf{central} ou \textbf{descentralizada}; com 
percepção \textbf{local} ou \textbf{global}; com tomada de decisão \textbf{
aleatória} ou \textbf{orientada a objetivo}. Os agentes reativos são aqueles que 
agem baseados apenas na sua percepção atual do território, enquanto que os 
agentes cognitivos podem perseguir um determinado objetivo. Enquanto os agentes 
cognitivos tem uma visão do grafo completo, os reativos só podem enxergar os nós 
adjacentes ao que eles se encontram. A comunicação se refere a possibilidade dos 
agentes trocarem informações. A percepção se refere ao quanto de informação o 
agente pode acessar sobre o ambiente ao seu redor e sobre os outros agentes. 
Finalmente, no esquema de coordenação centralizado, uma entidade central escolhe 
nó-objetivo de cada agente, enquanto que no esquema descentralizado, a 
coordenação emerge da interação entre os agentes.

As abordagens propostas na presente pesquisa podem ser classificadas como 
estratégias \textbf{baseadas em particionamento}, com agentes 
\textbf{cognitivos}, de comunicação \textbf{proibida}, com coordenação 
\textbf{centralizada}, de percepção \textbf{global}, e com tomada de decisão 
\textbf{orientada a objetivo}.