\chapter{Trabalhos Relacionados}
\label{chp:background}

Neste capítulo foram revisitados alguns trabalhos realizados sobre a \ac{tmap}. 
Primeiramente, algumas considerações foram feitas sobre os diferentes nomes,  
escopos e métricas utilizados para a \ac{tmap} pelos autores. Depois, foi 
realizada uma revisão de duas das definições formais mais utilizadas para o 
problema. Em seguida, foram apresentadas as classificações para estratégias 
que visam resolver a \ac{tmap}. Finalmente, as soluções já presentes na 
literatura foram analisadas.

\section{Considerações Iniciais}
\label{sec:consideracoesIniciais}

Um dos grandes obstáculos para o estudo do Problema da Patrulha Multiagente é a 
falta de concordância, dentre os pesquisadores, sobre nomenclatura, escopo e 
critérios de avaliação para as soluções propostas. Isso pode se dever ao fato do 
problema da patrulha estar presente em diferentes áreas como Inteligência 
Artificial, Sistemas Multiagentes e até Robótica \citep{sampaiophd}.

Há, na literatura, nomes diferentes para o problema da Patrulha Multiagente. Por 
exemplo, alguns autores \citep{hernandez2013game} chamam o problema de “patrulha 
multirobô” (ou Multi-Robot Patrolling, em inglês), outros \citep{6495145} se 
referem ao problema pelo nome de “patrulha temporal” (tradução de timed 
patrolling), já alguns pesquisadores \citep{Koenig:2001:TCA:375735.376463} 
utilizam o termo “cobertura de terreno” (terrain coverage, em inglês). No 
entanto, no decorrer do presente trabalho será utilizado o termo “Patrulha 
Multiagente”, pois ele é considerado mais apropriado por diversos pesquisadores 
\citep{6900280}, \citep{sampaiophd}, e \citep{6315145}.

O escopo é outro fator que varia entre os trabalhos. \citep{6615158}, por 
exemplo, faz uma análise do problema da patrulha onde o ambiente é dinâmico. 
Isto é, o grafo onde os agentes patrulham sofre alterações ao longo da execução 
do agente. Em um dado momento, o grafo pode ter vértices adicionados ou 
removidos. Já \citep{6900280} trabalham levando em conta que agentes podem ter 
velocidades diferentes. \citep{Pippin:2013:PBT:2480362.2480378} 
levam em consideração que agentes podem agir com eficiência abaixo do esperado, 
isto é, o agente não pode ser confiado para realizar a tarefa que lhe foi 
passada sem falhas. Em \citep{4209122}, os autores consideram restrições de 
frequência, isso significa que cada nó tem a si designado um valor de frequência 
com a qual o nó deve ser visitado. Outro exemplo seria o trabalho feito 
em \citep{6495145} e \citep{Poulet:2012:b}, onde é feita uma análise do problema 
em uma configuração de sistema aberto. Sistemas abertos foram definidos em 
\citep{6040660} como aqueles onde agentes podem entrar e sair a qualquer momento 
da execução.

Na presente pesquisa, será utilizado o escopo para a Patrulha Multiagente 
doravante referido como "padrão", onde os agentes possuem eficiência idêntica, 
são confiáveis e não são retirados nem adicionados ao longo da patrulha. Quanto 
ao ambiente, o objeto de estudo deste trabalho compreende apenas ambientes que 
permanecem estáticos durante a execução dos agentes, e os pontos que devem ser 
patrulhados não possuem restrições de frequência.

Existem diversas métricas que podem ser utilizadas para medir a eficiência de 
cada solução para a \ac{tmap}. Diferentes pesquisadores utilizam métricas 
distintas. Em \citep{Machado:2002:MPE:1765317.1765332} foram propostas as 
métricas mais utilizadas para a \ac{tmap}. São elas: ociosidade instantânea do 
nó, ociosidade instantânea do grafo, ociosidade máxima e tempo de exploração. 
Depois deste trabalho, \citep{sampaiophd} propôs uma nova família de métricas 
baseadas nos intervalos entre visitas. As métricas utilizadas nos trabalhos mais 
recentes variam: \citep{6900280} e \citep{Pippin:2013:PBT:2480362.2480378} 
utilizam o intervalo máximo, já \citep{4209122} fazem uso de três métricas de 
frequência: média, mínima e desvio padrão, enquanto que 
\citep{hernandez2013game} compara as frequências mínimas.

Dessa forma, este capítulo visa apresentar os trabalhos desenvolvidos para o 
problema da patrulha presentes na literatura classificando-as por seu escopo e 
métricas utilizadas na avaliação dos agentes.

\section{Levantamento Bibliográfico}

\subsection{Definições}
\label{definicoes_tmap}

O problema da Patrulha é tipicamente modelado por um Grafo 
\citep{Rosen:2002:DMA:579402}. Segundo \citep{Almeida:2004:AAI}, isso se deve 
ao fato de que representar o problema por meio de um grafo faz com que o 
problema possa ser facilmente adaptado para um variedade de domínios desde 
terrenos até redes de computadores. \citep{sampaiophd} também considera que os 
grafos sejam o modelo preferencial para os ambientes da \ac{tmap}, pois são 
suficientemente poderosos para capturar as características do terreno mais 
relevantes para o problema. Essa capacidade do grafo pode ser confirmada por 
estudos que adotam grafos como modelo e posteriormente aplicam suas soluções em 
ambientes realistas contínuos \citep{Pippin:2013:PBT:2480362.2480378}.

Em \citep{Chevaleyre:2004:TAM:1018411.1019013}, o problema é matematicamente 
definido da seguinte forma:

O território onde os agentes patrulham é representado por um grafo $G(V,E)$, 
onde $V$ é o conjunto dos pontos de interesse que precisam ser patrulhados e 
$E$ representa o conjunto de arestas, $E \subset V^{2}$, de $G$. Para toda 
aresta $(i,j) \in E$, onde $i,j \in V$, corresponde um peso $c_{i,j}$ 
representando a distância entre o vértices $i$ e $j$ em $G$. 

Nesse cenário, uma solução monoagente para o problema seria uma função 
$ \pi : \mathbb{N} \longrightarrow V$, tal que $ \pi(j)$ 
é o $j$-ésimo vértice visitado pelo agente, desde que 
$ \pi(j+1) = x $ somente se $ (\pi(j), x) \in E $. Analogamente, 
uma solução multiagente seria um conjunto $ \Pi = \{ \pi_{1} ... \pi_{r} \}$, 
onde $r$ é o número de agentes.

Dessa forma, o problema seria encontrar o conjunto $ \Pi $ que obtivesse os 
melhores resultados de acordo com um certo critério de avaliação, como por 
exemplo, o intervalo máximo das visitas nos pontos de interesse.

Já \citep{sampaiophd} contribuiu com uma definição mais abrangente da \ac{tmap}. 
Ele aponta que uma instância da \ac{tmap} pode ser completamente definida pela 
seguinte tupla: 

$$ \langle E, P, S, s_{0}, A, I, M \rangle $$

O elemento $E$ representa o ambiente (do inglês, \textit{environment} onde estão 
os pontos de interesse a serem patrulhados, preferencialmente $E$ deve ser um 
grafo $G(V,E)$. O elemento $P$ é o conjunto dos pontos de interesse do ambiente. 
No caso de um grafo, $P = V$.

$S$ é um elemento um pouco mais complexo, pois ele representa o conjunto de 
possíveis estados da sociedade de agentes. Cada estado dentro desse conjunto 
pode conter informações tais como: quantos agentes estão ativos, qual a posição 
atual de cada agente no ambiente $E$, o tempo decorrido desde o início da 
patrulha, orientação e energia de cada agente e características globais do 
conjunto de agentes. O estado inicial da sociedade é representado em $s_{0}$, ou 
seja, $s_{0} \in S$. Esta modelagem para o grupo de agentes permite ao modelo de 
\citep{sampaiophd} englobar diversos escopos da \ac{tmap}, como por exemplo os 
citados na \secref{sec:consideracoesIniciais}.

O quinto elemento da tupla, $A$, é o conjunto de ações que alteram o estado da 
sociedade, $S$. Essas ações podem ser individuais, de cada agente, ou coletivas. 
O autor afirma que $A$ deve conter, no mínimo, duas ações individuais definidas 
para cada agente: movimentação entre os pontos de $P$ e visitação aos elementos 
do conjunto.

$I$ representa o intervalo de medição, dentro do qual o desempenho dos agentes 
é mensurado através da métrica, $M$, último elemento da 
tupla.

O problema da patrulha multiagente seria, então, definir um conjunto $A$ de 
ações a serem tomadas pelos agentes, que estão inicialmente no estado $s_{0}$, 
para minimizar (ou maximizar) a métrica, $M$, durante o período de patrulha, 
$I$.

Uma vez que a \ac{tmap} está bem definida, a seção seguinte irá abordar as 
diferentes formas de classificar soluções para a \ac{tmap}, de forma que ao 
final, as abordagens a serem propostas neste trabalho podem ser classificadas.

\subsection{Classificações para a \ac{tmap}}
\label{sec:classifytmap}

São notórios alguns trabalhos muito citados na literatura que visaram 
classificar as soluções para a \ac{tmap}. Em 
\citep{Chevaleyre:2004:TAM:1018411.1019013}, as abordagens propostas até 
então foram classificadas em cíclicas (ou de ciclo único) e 
baseadas em particionamento. As soluções de ciclo único são aquelas onde é 
calculado um ciclo que cobre todos os vértices do grafo, e então, os agentes são 
colocados para caminhar nesse ciclo indefinidamente. As abordagens baseadas em 
particionamento são aquelas onde o território a ser patrulhado é dividido em 
$r$ regiões, onde $r$ é o número de agentes. Os agentes devem, então, patrulhar 
dentro de suas respectivas regiões.

Já \citep{Machado:2002:MPE:1765317.1765332} faz uma extensa classificação das 
arquiteturas de soluções para a \ac{tmap}. O trabalho divide os agentes em: 
reativo ou cognitivo; com comunicação permitida ou 
proibida; com coordenação central ou descentralizada; com 
percepção local ou global; com tomada de decisão 
aleatória ou orientada a objetivo. Os agentes reativos são aqueles que 
agem baseados apenas na sua percepção atual do território, enquanto que os 
agentes cognitivos podem perseguir um determinado objetivo. Enquanto os agentes 
cognitivos tem uma visão do grafo completo, os reativos só podem enxergar os nós 
adjacentes ao que eles se encontram. A comunicação se refere a possibilidade dos 
agentes trocarem informações. A percepção se refere ao quanto de informação o 
agente pode acessar sobre o ambiente ao seu redor e sobre os outros agentes. 
Finalmente, no esquema de coordenação centralizado, uma entidade central escolhe 
nó-objetivo de cada agente, enquanto que no esquema descentralizado, a 
coordenação emerge da interação entre os agentes.

Com o problema bem descrito, pode-se explorar as soluções que já estão presentes 
na literatura e verificar se há espaço para a proposição de mais estratégias 
para a \ac{tmap}.

\subsection{Soluções já propostas}

\citep{Chevaleyre:2004:TAM:1018411.1019013} propuseram uma solução para a 
\ac{tmap} baseada no \ac{tsp}. Segundo os autores, a solução mais simples seria 
encontrar um ciclo que percorresse todos os pontos de interesse do terreno 
(de preferência sem repetição de pontos dentro do ciclo) e então fazer o agente 
percorrer esse ciclo indefinidamente. Esta estratégia corresponde a encontrar um 
ciclo que percorre todos os nós do grafo sem repeti-los, um Ciclo Hamiltoniano. 
O trabalho demonstra que para um único agente a solução ótima, para a métrica da 
ociosidade máxima, é esta estratégia cíclica baseada no \ac{tsp}. No entanto, 
o \ac{tsp} é conhecidamente um problema NP-Completo. Dessa forma, os autores 
propõem uma solução utilizando o algoritmo de Christofides 
\citep{christofides1976worst} para obter uma aproximação do \ac{tsp} em tempo 
polinomial. A pesquisa ainda aponta que, para o caso multiagente, a solução 
seria calcular o ciclo, posicionar os agentes com um certo intervalo entre um e 
outro e colocá-los para percorrer o ciclo indefinidamente. Contudo, essa 
abordagem para o caso multiagente não é ótima. Os autores, no entanto, não 
estudam a aplicação dessa solução ao problema com outras métricas que não a da 
ociosidade máxima.

Posteriormente, \citep{Almeida:2004:AAI} realizam um estudo comparativo entre 
estratégias centralizadas e descentralizadas. Eles concluem que a solução de 
ciclo único proposta por \citep{Chevaleyre:2004:TAM:1018411.1019013} é a mais 
eficiente dentre as estudadas por eles.

A partir das conclusões tiradas nesses estudos, \citep{4209122} propõem uma 
solução centralizada e cíclica, onde ao invés de utilizar o algoritmo de 
Christofides, se utiliza o método \textit{Spanning Tree Coverage}, introduzido 
em \citep{Gabriely:2001}, para gerar um Ciclo Hamiltoniano em um grafo com 
formato de \textit{grid}. Assim como a solução de 
\citep{Chevaleyre:2004:TAM:1018411.1019013}, após o ciclo ser calculado, os 
vários agentes são posicionados no grafo e colocados para percorrer o ciclo 
indefinidamente.

Além dessas soluções de ciclo único, também foram propostas, na literatura, 
soluções baseadas em particionamento. Os autores 
\citep{Pippin:2013:PBT:2480362.2480378} propõem uma solução, onde, na 
inicialização, os nós do grafo são particionados entre os agentes e esses 
calculam um ciclo fechado (\textit{closed cycle}) dentro da respectiva partição. 
Uma entidade de monitoramento central fica responsável por verificar, ao longo 
da execução, a performance dos agentes para a métrica de intervalo máximo entre 
visitas e identificar agentes que estejam com baixo desempenho. Então, alguns 
nós do agente de baixo rendimento são oferecidos aos outros agentes e 
redesignados através de um protocolo baseado em leilão.

Visando a escalabilidade apontada por \citep{Almeida:2004:AAI}, foram propostas 
diversas soluções descentralizadas. Dentre elas, destacam-se as baseadas em 
colônias de formigas \citep{Koenig:2001:TCA:375735.376463}, 
\citep{Elor:2010:AMC:1884958.1884970}, \citep{6615158}. Nessas soluções, os 
agentes não possuem uma visão completa do grafo, mas apenas das suas 
vizinhanças. A coordenação dos agentes é emergente 
\citep{Machado:2002:MPE:1765317.1765332}, pois cada agente pode deixar marcações 
em nós ou arestas do grafo que podem ser sentidas pelos outros agentes.

Apesar de serem amplamente utilizados na literatura para resolver problemas de 
otimização, os Algoritmos Evolucionários \citep{Luke2013Metaheuristics} foram 
pouco utilizados no problema da Patrulha. \citep{4630897} propõe uma solução 
híbrida composta de dois passos: no primeiro, um algoritmo genético para 
posicionar os agentes o mais distante possível dentro do território, e então, 
utiliza \ac{aco} para calcular partições no grafo por onde os agentes irão 
patrulhar durante a execução. Dessa forma, pode-se constatar que o algoritmo 
evolucionário não está sendo usado para resolver o problema da patrulha em si, 
mas para encontrar os $r$ nós mais distantes uns dos outros dentro do grafo, 
onde $r$ é o número de agentes. Depois, \citep{6900280} utiliza a mesma 
abordagem híbrida num escopo onde os agente se movem com velocidades distintas e 
os nós possuem prioridades diferentes, além de apresentar uma variação 
\textit{multicore} da solução.

\section{Considerações Finais}

Uma vez apresentados, neste capítulo, os conceitos básicos dos trabalhos 
relacionados, é possível fazer algumas observações sobre tendências dentre as 
publicações. Há uma predileção pela Simulação como método de comparação e de 
demonstração da relevância das soluções. É possível também perceber que muitos 
trabalhos utilizam as métricas propostas por 
\citep{Machado:2002:MPE:1765317.1765332} e \citep{sampaiophd}.

Os trabalhos comparativos feitos por \citep{Almeida:2004:AAI} e 
\citep{sampaiophd} mostram que não existem estratégias claramente dominantes 
ou ótimas para \ac{tmap}. Percebe-se também que falta à literatura um trabalho 
que investigue uma solução para a \ac{tmap} utilizando algoritmos 
evolucionários, tais como as Estratégias Evolucionárias (ES)\acused{es} e os Algoritmos Genéticos\acused{ga}. Dessa forma, nos próximos 
capítulos, serão apresentados alguns algoritmos evolucionários que podem ser 
utilizados para resolver a \ac{tmap}, em seguida, serão propostas formas de 
aplicar esses algoritmos no contexto do problema e, finalmente, será 
demonstrada a eficácia dessas estratégias perante as abordagens de destaque 
propostas por outros autores.

O \tabref{tbl:comp_background} resume as soluções apresentadas nesta sessão de 
acordo com o escopo do problema, o método de demonstração das soluções e as 
métricas utilizadas.

\begin{table}[tp]
	\centering
	\caption{Resumo dos trabalhos relacionados}
	\label{tbl:comp_background}
	\begin{tabularx}{\linewidth}{|X|X|X|X|}
		\hline
		\textbf{Trabalho} & \textbf{Escopo} & \textbf{Método} & \textbf{Métrica(s)} \\
		\hline
		\citep{Machado:2002:MPE:1765317.1765332} & Patrulha Multiagente "Padrão" & Simulação & Ociosidade do grafo, ociosidade máxima e tempo de exploração \\
		\hline
		\citep{Chevaleyre:2004:TAM:1018411.1019013} & Patrulha Multiagente "Padrão" & Análise Matemática & Pior ociosidade (\textit{worst idleness}) \\
		\hline
		\citep{Almeida:2004:AAI} & Patrulha Multiagente "Padrão" & Simulação & Ociosidade do grafo, Ociosidade média normalizada e Desvio padrão da ociosidade média \\
		\hline
		\citep{4209122} & Patrulha Multiagente com restrição de frequência & Análise Matemática & Frequência Mínima, Frequência Média e Desvio padrão das frequências das visitas \\
		\hline
		\citep{4630897} & Patrulha Multiagente "Padrão" & Simulação & Ociosidade Máxima \\
		\hline
		\citep{6495145} & Patrulha Multiagente como um sistema aberto & Simulação & Intervalo médio entre visitas, Intervalo quadrático médio, tempo de estabilização e amplitude das variações \\
		\hline
		\citep{Pippin:2013:PBT:2480362.2480378} & Patrulha Multiagente com agentes que possuem performances distintas & Simulação e Experimentação com robôs reais & Intervalo Máximo entre visitas \\
		\hline
		\citep{6615158} & Patrulha Multiagente em ambientes dinâmicos & Simulação & Intervalo Máximo entre visitas \\
		\hline
		\citep{hernandez2013game} & Patrulha Multiagente "Padrão" & Simulação & Ociosidade média \\
		\hline
	\end{tabularx}
	\caption*{Fonte: O autor}
\end{table}