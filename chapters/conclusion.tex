\chapter{Conclusão}

Como mostrado neste documento, a Patrulha Multiagente Temporal tem sido o objeto 
de estudo de diversas pesquisas, pois trata-se de um problema difícil e 
desafiador da área de Inteligência Artificial e Sistemas Multiagente com 
diversas aplicações em cenários reais.

No entanto, também foi verificado, que apesar de ser essencialmente um problema 
de otimização, não haviam sido estudadas aplicações diretas de algoritmos 
evolucionários para encontrar soluções para a \ac{tmap}.

Assim, após revisar quatro algoritmos bastante utilizados na Computação 
Evolucionária, a presente pesquisa apresentou dez operadores que podem ser 
combinados com esses algoritmos de diferentes maneiras para formar dezenas de 
heurísticas evolucionárias distintas capazes de resolver a \ac{tmap}.

Além disso, foi demonstrado através de experimentos que os operadores aqui 
apresentados são eficazes em produzir boas soluções. Também foram apontados 
os operadores e suas combinações que proporcionaram as melhores soluções. 
E foi mostrado empiricamente que estas heurísticas evolucionárias não deixam 
a desejar em questão de qualidade das soluções encontradas quando comparadas 
com outras estratégias já apresentadas na literatura.

\section{Trabalhos Futuros}

Este capítulo conclui o trabalho com algumas atividades que poderiam ter sido 
desenvolvidas, não fosse o período de tempo curto disponibilizado para sua 
conclusão.

Foi observado durante os experimentos que os operadores de mutação geravam 
indivíduos novos diferentes o bastante dos pais para permitir encontrar novas 
soluções e evoluir a população. No entanto, foi possível perceber ao longo dos 
experimentos realizados que esses operadores geravam indivíduos muito próximos 
uns dos outros. Fica como trabalho futuro investigar operadores de mutação mais 
drásticos, que levem a uma diversidade maior da população, para que a busca no 
espaço de soluções seja mais global e não fique "presa" ao redor de mínimos 
locais.

Outra contribuição para o futuro seria investigar o impacto de outros operadores 
de seleção nos algoritmos genéticos. Principalmente o algoritmo genético de 
estado estável que utiliza duas seleções distintas: uma para escolher os 
indivíduos que deverão sofrer recombinação e outra para escolher os indivíduos 
da geração mais velha que irão "morrer" para dar lugar às novas soluções 
recém-descobertas por meio da mutação e recombinação. Na presente pesquisa, 
esta última seleção foi configurada como Seleção da Pior Aptidão, ou seja, os 
indivíduos menos aptos seriam escolhidos para dar lugar. Uma outra seleção 
que pudesse eventualmente selecionar indivíduos bastante aptos, poderia fazer 
com que a busca do algoritmo genético de estado estável escapasse de mínimos 
locais em volta desses.

Nos experimentos de \textit{tuning}, os operadores puramente aleatórios, como 
por exemplo, o \textit{Random Centering} e \textit{Random Partitioning}, tiveram 
um desempenho bem abaixo dos seus concorrentes. Seria de grande valia investigar 
se, dado mais tempo de execução para os algoritmos evolucionários, os operadores 
puramente aleatórios poderiam ser vantajosos, proporcionando uma busca mais 
global para os algoritmos.