\chapter{Experimentos}
\label{experimentos}

Os operadores propostos nesta pesquisa podem ser compostos de diferentes formas, 
construindo diversas heurísticas que podem encontrar soluções distintas para 
\ac{tmap}. Assim, este capítulo tem dois objetivos. O primeiro é encontrar as 
melhores combinações de operadores. O segundo é comparar as heurísticas que 
obtiveram melhores resultados com abordagens propostas por outros autores.

Os operadores descritos neste trabalho foram implementados na linguagem de 
programação Java\footnote{\url{http://java.com/pt_BR/}}. A principal razão para 
a adoção desta linguagem foi o \textit{Simple Patrol}, simulador da \ac{tmap}, 
desenvolvido pelo grupo de pesquisa sobre Patrulha Multiagente da Universidade 
Federal Rural de Pernambuco, utilizado para realizar os experimentos desta 
pesquisa. Os operadores foram desenvolvidos de forma compatível com a biblioteca 
jMetal\footnote{\url{http://jmetal.sourceforge.net/}} \citep{Durillo2011}, que 
disponibiliza diversos algoritmos evolucionários facilmente adaptáveis para 
qualquer problema que possa ser modelado em classes Java.

Uma vez que os operadores estavam implementados e importados no simulador da 
\ac{tmap}, foram realizados diversos experimentos divididos em dois grupos. O 
primeiro com o objetivo de encontrar as melhores abordagens evolucionárias para 
\ac{tmap} e o segundo com a finalidade de comparar estas abordagens com as 
publicadas por outros autores.

Antes de descrever esses experimentos, serão discutidos os parâmetros que podem 
ser variados para gerar heurísticas evolucionárias distintas.

\begin{itemize}
	\item A Métrica utilizada para avaliar os indivíduos se manteve constante em 
	todos os experimentos. A média quadrática dos intervalos foi utilizada. 
	%citar pablo e explicar porque essa métrica é uma boa ideia
	\item O número máximo de avaliações
\end{itemize}

\section{\textit{Tuning}}

\textit{Tuning} pode ser traduzido, do inglês, para ajustar. Esta seção tem como 
finalidade descrever e apresentar os resultados dos experimentos desenvolvidos 
para buscar o ajuste do operadores que compõem as melhores heurísticas 
evolucionárias encontradas neste estudo. Por exemplo, qual seria a melhor forma 
de inicializar um indivíduo? Existem ao todo doze formas diferentes de compor 
o operador de inicialização de indivíduos de uma heurística evolucionária 
utilizando os operadores propostos neste trabalho.

Primeiramente, foram testados os tamanhos de população. Nos algoritmos genéticos, 
esse parâmetro é diz respeito apenas a quantidade de indivíduos a cada geração. 
Já nas estratégias evolucionárias existem dois parâmetros relacionados ao tamanho 
da população: $\mu$ e $\lambda$, como foi discorrido no \chapref{alg_evo}.