\chapter{Algoritmos Evolucionários}

Sendo o objetivo do presente trabalho apresentar abordagens evolucionárias para 
a \ac{tmap}, este capítulo tem como finalidade apresentar ao leitor alguns 
conceitos e terminologias da área da Computação Evolucionária, para que ele 
esteja familiarizado antes do capítulo sobre a aplicação de algoritmos 
evolucionários no contexto da \ac{tmap}. Este capítulo também vai apresentar os 
algoritmos que foram utilizados na pesquisa.

Segundo \citep{Back:1993:OEA:1326623.1326625}, várias pesquisas mostraram, ao 
longo de três séculos, que modelar o processo de busca de forma similar à evolução 
pelo qual os seres vivos passaram pode render algoritmos robustos, mesmo que 
estes modelos sejam apenas representações imperfeitas do verdadeiro processo 
biológico. O resultado desses modelos são chamados de Algoritmos Evolutivos ou 
Evolucionários. Essa busca pode ser aplicada para encontrar não apenas uma 
solução qualquer, mas aquela que minimize ou maximize uma dada métrica. Dessa 
forma, algoritmos evolucionários podem (e são) utilizados em problemas de 
otimização.

Tais algoritmos são baseados no processo de aprendizado coletivo pelo qual passa 
uma população de indivíduos. Cada indivíduo representa uma solução para um 
problema, ou um ponto no espaço de possíveis soluções. O ambiente fornece 
informações qualitativas sobre cada indivíduo. Essa informação pode ser 
interpretada como a aptidão do indivíduo no ambiente determinado. Na 
analogia com algoritmos, essa é a medida de avaliação de uma solução 
\citep{Back:1993:OEA:1326623.1326625}.

Um algoritmo evolucionário funciona, genericamente, da seguinte forma: uma 
população inicial é arbitrariamente inicializada; esse indivíduos têm sua 
adaptação ao ambiente medida; eles são, posteriormente, recombinados para formar 
uma nova população, podendo também sofrer mutação; finalmente, um subconjunto 
dessas populações (antiga e nova, pais e filhos) é selecionado de alguma forma 
definida pelo algoritmo e se torna a população da próxima geração. Esse ciclo 
se repete tipicamente até que parem de surgir melhores indivíduos que aqueles já 
presentes na população, evento chamado de convergência do algoritmo 
\citep{Back:1993:OEA:1326623.1326625}.

A população inicial pode ser criada de forma aleatória, ou pode-se utilizar 
conhecimentos sobre o problema para inicializar uma população onde os indivíduos 
estão em uma região "boa" (de acordo com a métrica) do espaço de soluções 
\citep{Luke2013Metaheuristics}.

O Pseudocódigo~\ref{EAGeneric} abaixo exemplifica de forma genérica um algoritmo 
evolucionário.

\begin{algorithm}                      % enter the algorithm environment
	\caption{Algoritmo Evolucionário Genérico}          % give the algorithm a caption
	\label{EAGeneric}                           % and a label for \ref{} commands later in the document
	\begin{algorithmic}                    % enter the algorithmic environment
		\Procedure{EA}{}
		\State $P \gets $ Constrói-População-Inicial
		\Repeat
			\State $P^{\prime} \gets $ Recombina($P$)
			\State $P^{\prime \prime} \gets $ Aplica-Mutação($P^{\prime}$)
			\State $P \gets $ Seleciona($P^{\prime \prime}$)
		\Until{não termos mais tempo}
		\EndProcedure
	\end{algorithmic}
\end{algorithm}

A \tabref{tbl:evo_dict} revisa alguns dos termos comumente utilizados nos 
\acp{ea}.

No restante do presente capítulo, serão apresentados os algoritmos utilizados 
nesta pesquisa, as Estratégias Evolucionárias e os Algoritmos Genéticos. Cada 
um desses possui duas variações que foram aplicadas na pesquisa: 
$( \mu,\lambda )$ ou $( \mu + \lambda )$ e Geracional ou Estado Estável 
(em inglês, \textit{steady state}), respectivamente.

\begin{table}[tp]
	\centering
	\caption{Termos comuns na Computação Evolucionária}
	\label{tbl:evo_dict}
	\begin{tabularx}{\linewidth}{|X|X|}
		\hline
		\textbf{Termo} & \textbf{Significado} \\
		\hline
		Indivíduo & Uma solução candidata \\
		\hline
		Filha e Pai & Uma solução Filha é uma cópia alterada de uma solução Pai \\
		\hline
		População & Um conjunto de soluções candidatas \\
		\hline
		Aptidão & Qualidade de um indivíduo (Solução) \\
		\hline
		Seleção & Coletar indivíduos baseados na sua aptidão \\
		\hline
		Mutação & Realização de pequenas alterações em indivíduos. Também chamada de reprodução assexuada \\
		\hline
		Recombinação & Uma forma de alteração especial que recebe como parâmetro duas soluções pais e (normalmente) produz duas soluções filhas \\
		\hline
		Reprodução & O ato de produzir uma ou mais soluções filhas a partir de uma solução pai \\
		\hline
		Geração & Um ciclo de medida de aptidão ou de reprodução de uma população \\
		\hline
%		\citep{Machado:2002:MPE:1765317.1765332} & Patrulha Multiagente "Padrão" & Simulação & Ociosidade instantânea do nó, ociosidade instantânea do grafo, ociosidade do grafo, ociosidade máxima e tempo de exploração \\
%		\hline
	\end{tabularx}
	\caption*{Fonte: Adaptado de \citep{Luke2013Metaheuristics}}
\end{table}

\subsection{Estratégias Evolucionárias}

As duas estratégias evolucionárias usadas no presente trabalho se diferem pela 
forma como fazem a composição entre a população de pais e a população de filhos 
para construir a nova geração, que será utilizada na iteração seguinte do 
algoritmo.

A primeira \ac{es} é conhecida como $( \mu, \lambda )$. Tipicamente, começa-se 
com uma população de \lambda indivíduos gerados de forma arbitrária. Nessa 
\ac{es}, o \mu representa o número de pais cujos filhos serão usados para compor 
a próxima população que também deve ter \lambda indivíduos no total. Note que, 
então, \lambda tem que ser um múltiplo de \mu. Então, os \mu indivíduos mais 
aptos são escolhidos, processo chamado de Seleção por Truncamento (em inglês, 
\textit{Truncate Selection}. Os indivíduos selecionados sofreram mutação para 
gerar $ \lambda \/ \mu $ filhos. O que acarretará em uma nova população de 
\lambda indivíduos que será a geração utilizada na próxima iteração do 
algoritmo. O Pseudocódigo~\ref{mu_lambda_es} exemplifica a Estratégia 
Evolucionária $( \mu, \lambda )$.

\begin{algorithm}                      % enter the algorithm environment
	\caption{$( \mu, \lambda )$ Estratégia Evolucionária}          % give the algorithm a caption
	\label{mu_lambda_es}                           % and a label for \ref{} commands later in the document
	\begin{algorithmic}                    % enter the algorithmic environment
		\Procedure{$( \mu, \lambda )$ ES}{$\lambda, \mu$}
		\State $P \gets \{\} $
		\For{\lambda times}
			\State $P \gets P \cup $ \{novo indivíduo gerado de forma arbitrária\}
		\EndFor
		\State $Melhor \get $ nulo
		\Repeat
			\For{$P_{i} \in P$}
				CalculaAptidão($P_{i}$)
				\If{$Melhor = $ nulo ou Aptidão($P_{i}$) > Aptidão($Melhor$)}
					$Melhor \gets P_{i}$
				\EndIf
			\EndFor
		\Until{não termos mais tempo}
		\EndProcedure
	\end{algorithmic}
\end{algorithm}