\chapter{Algoritmos Evolucionários}

Segundo \citep{Back:1993:OEA:1326623.1326625}, várias pesquisas mostraram, ao 
longo de três séculos, que modelar o processo de busca de forma similar à evolução 
pelo qual os seres vivos passaram pode render algoritmos robustos, mesmo que 
estes modelos sejam apenas representações imperfeitas do verdadeiro processo 
biológico. O resultado desses modelos são chamados de Algoritmos Evolutivos ou 
Evolucionários.

Tais algoritmos são baseados no processo de aprendizado coletivo pelo qual passa 
uma população de indivíduos. Cada indivíduo representa uma solução para um 
problema, ou um ponto no espaço de possíveis soluções. O ambiente fornece 
informações qualitativas sobre cada indivíduo. Essa informação pode ser 
interpretada como a aptidão do indivíduo no ambiente determinado. Na 
analogia com algoritmos, essa é a medida de avaliação de uma solução 
\citep{Back:1993:OEA:1326623.1326625}.

Um algoritmo evolucionário funciona, genericamente, da seguinte forma: uma 
população inicial é arbitrariamente inicializada; esse indivíduos têm sua 
adaptação ao ambiente medida; eles são, posteriormente, recombinados para formar 
uma nova população, podendo também sofrer mutação; finalmente, um subconjunto 
dessas populações é selecionado de alguma forma definida pelo algoritmo e se 
torna a população da próxima geração. Esse ciclo se repete tipicamente até que 
parem de surgir melhores indivíduos que os já presentes na população, evento 
chamado de convergência do algoritmo \citep{Back:1993:OEA:1326623.1326625}.

O Algoritmo~\ref{EAGeneric} abaixo exemplifica de forma genérica um algoritmo evolucionário.

\begin{algorithm}                      % enter the algorithm environment
	\caption{Algoritmo Evolucionário Genérico}          % give the algorithm a caption
	\label{EAGeneric}                           % and a label for \ref{} commands later in the document
	\begin{algorithmic}                    % enter the algorithmic environment
		\Procedure{EA}{}
		\State $P \gets $ Constrói-População-Inicial
		\Repeat
			\State $P^{\prime} \gets $ Recombina($P$)
			\State $P^{\prime \prime} \gets $ Aplica-Mutação($P^{\prime}$)
			\State $P \gets $ Seleciona($P^{\prime \prime}$)
		\Until{não termos mais tempo}
		\EndProcedure
	\end{algorithmic}
\end{algorithm}

A \tabref{tbl:evo_dict} revisa alguns dos termos comumente utilizados nos 
\acp{ea}.

No restante do presente capítulo, serão apresentados os algoritmos utilizados 
nesta pesquisa, as Estratégias Evolucionárias e os Algoritmos Genéticos.

\begin{table}[tp]
	\centering
	\caption{Termos comuns na Computação Evolucionária}
	\label{tbl:evo_dict}
	\begin{tabularx}{\linewidth}{|X|X|}
		\hline
		\textbf{Termo} & \textbf{Significado} \\
		\hline
		Indivíduo & Uma solução candidata \\
		\hline
		Filha e Pai & Uma solução Filha é uma cópia alterada de uma solução Pai \\
		\hline
		População & Um conjunto de soluções candidatas \\
		\hline
		Aptidão & Qualidade de um indivíduo (Solução) \\
		\hline
		Seleção & Coletar indivíduos baseados na sua aptidão \\
		\hline
		Mutação & Realização de pequenas alterações em indivíduos. Também chamada de reprodução assexuada \\
		\hline
		Recombinação & Uma forma de alteração especial que recebe como parâmetro duas soluções pais e (normalmente) produz duas soluções filhas \\
		\hline
		Reprodução & O ato de produzir uma ou mais soluções filhas a partir de uma solução pai \\
		\hline
		Geração & Um ciclo de medida de aptidão ou de reprodução de uma população \\
		\hline
%		\citep{Machado:2002:MPE:1765317.1765332} & Patrulha Multiagente "Padrão" & Simulação & Ociosidade instantânea do nó, ociosidade instantânea do grafo, ociosidade do grafo, ociosidade máxima e tempo de exploração \\
%		\hline
	\end{tabularx}
	\caption*{Fonte: Adaptado de \citep{Luke2013Metaheuristics}}
\end{table}

\subsection{Estratégias Evolucionárias}

