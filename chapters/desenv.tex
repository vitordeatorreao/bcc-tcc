\chapter{Algoritmos Evolucionários}

Segundo \citep{Back:1993:OEA:1326623.1326625}, várias pesquisas mostraram, ao 
longo de três séculos, que modelar o processo de busca de forma similar à evolução 
pelo qual os seres vivos passaram pode render algoritmos robustos, mesmo que 
estes modelos sejam apenas representações imperfeitas do verdadeiro processo 
biológico. O resultado desses modelos são chamados de Algoritmos Evolutivos ou 
Evolucionários.

Tais algoritmos são baseados no processo de aprendizado coletivo pelo qual passa 
uma população de indivíduos. Cada indivíduo representa uma solução para um 
problema, ou um ponto no espaço de possíveis soluções. O ambiente fornece 
informações qualitativas sobre cada indivíduo. Essa informação pode ser 
interpretada como a medida do quão ajustado o indivíduo está no ambiente. Na 
analogia com algoritmos, essa é a medida de avaliação de uma solução 
\citep{Back:1993:OEA:1326623.1326625}.

Um algoritmo evolucionário funciona, genericamente, da seguinte forma: uma 
população inicial é arbitrariamente inicializada; esse indivíduos têm sua 
adaptação ao ambiente medida; eles são, posteriormente, recombinados para formar 
uma nova população, podendo também sofrer mutação; finalmente, um subconjunto 
dessas populações é selecionado de alguma forma definida pelo algoritmo e se 
torna a população da próxima geração. Esse ciclo se repete tipicamente até que 
parem de surgir melhores indivíduos que os já presentes na população, evento 
chamado de convergência do algoritmo \citep{Back:1993:OEA:1326623.1326625}.

O pseudocódigo abaixo exemplifica de forma genérica um algoritmo evolucionário.

A \tabref{tbl:evo_dict} fornece um dicionário de termos comumente utilizados nos 
\acp{ea}.