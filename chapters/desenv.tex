\chapter{Algoritmos Evolucionários}
\label{alg_evo}

Sendo o objetivo do presente trabalho aplicar abordagens evolucionárias para 
a \ac{tmap}, este capítulo tem como finalidade apresentar ao leitor alguns 
conceitos e terminologias da área da Computação Evolucionária, para que se 
familiarize antes do capítulo sobre a aplicação de algoritmos evolucionários no 
contexto da \ac{tmap}. Este capítulo também vai apresentar os algoritmos que 
foram utilizados como base para esta pesquisa.

Segundo \citep{Back:1993:OEA:1326623.1326625}, várias pesquisas mostraram que 
modelar o processo de busca de forma similar à evolução pelo qual os seres 
vivos passaram pode render algoritmos robustos, mesmo que estes modelos sejam 
apenas representações imperfeitas do verdadeiro processo biológico. O resultado 
desses modelos são chamados de Algoritmos Evolutivos ou Evolucionários. Essa 
busca pode ser aplicada para encontrar não apenas uma solução qualquer, mas 
aquela que minimize ou maximize uma dada métrica. Dessa forma, algoritmos 
evolucionários podem (e são) utilizados em problemas de otimização.

Tais algoritmos são baseados no processo de aprendizado coletivo pelo qual passa 
uma população de indivíduos. A cada geração, os indivíduos mais aptos passam 
para os filhos uma combinação das características que podem torná-los mais aptos. 
Então, ao longo das gerações, além dos indivíduos menos aptos serem extintos, 
novos indivíduos possivelmente mais aptos que os pais são formados por esse 
processo de reprodução \citep{Back:1993:OEA:1326623.1326625}.

Na Computação Evolutiva, os indivíduos são as soluções para um determinado 
problema, ou um ponto no espaço de possíveis soluções. Na \ac{tmap}, por 
exemplo, um indivíduo seria uma sociedade de agentes e suas trajetórias de 
patrulha. O ambiente seria o problema em si. Então, a aptidão de um indivíduo 
seria calculada através de uma medida de avaliação adequada ao problema.

Um algoritmo evolucionário funciona, genericamente, da seguinte forma: uma 
população inicial é arbitrariamente inicializada; esses indivíduos têm sua 
adaptação ao ambiente medida; eles são, posteriormente, recombinados para formar 
uma nova população, podendo também sofrer mutação; finalmente, um subconjunto 
dessas populações (antiga e nova, pais e filhos) é selecionado de alguma forma 
definida pelo algoritmo e se torna a população da próxima geração. Esse ciclo 
se repete tipicamente até que parem de surgir melhores indivíduos que aqueles já 
presentes na população, evento chamado de convergência do algoritmo 
\citep{Back:1993:OEA:1326623.1326625} ou até que algum outro critério de parada 
seja acionado.

A população inicial pode ser criada de forma aleatória, ou pode-se utilizar 
conhecimentos sobre o problema para inicializar uma população onde os indivíduos 
estão em uma região "boa" (de acordo com a métrica) do espaço de soluções 
\citep{Luke2013Metaheuristics}.

O Pseudocódigo~\ref{EAGeneric} abaixo exemplifica de forma genérica um algoritmo 
evolucionário.

\begin{algorithm}                      % enter the algorithm environment
	\caption{Algoritmo Evolucionário Genérico}          % give the algorithm a caption
	\algcomment{\begin{center} Fonte: Adaptado de \citep{Back:1993:OEA:1326623.1326625} \end{center}}
	\label{EAGeneric}                           % and a label for \ref{} commands later in the document
	\begin{algorithmic}                    % enter the algorithmic environment
		\Procedure{EA}{}
		\State $P \gets $ Constrói-População-Inicial
		\Repeat
			\State Avalia($P$)
			\State $P^{\prime} \gets $ Recombina($P$)
			\State $P^{\prime \prime} \gets $ Aplica-Mutação($P^{\prime}$)
			\State $P \gets $ Seleciona($P^{\prime \prime}$)
		\Until{não tenhamos mais tempo}
		\EndProcedure
	\end{algorithmic}
\end{algorithm}

O \tabref{tbl:evo_dict} revisa alguns dos termos comumente utilizados nos 
\acp{ea}.

No restante do presente capítulo, serão apresentados os algoritmos utilizados 
nesta pesquisa: as Estratégias Evolucionárias e os Algoritmos Genéticos. Este 
trabalho utilizou duas variações, propostas na literatura, para cada um deles 
que também serão apresentadas nas demais seções.
%Cada 
%um desses possui duas variações que foram aplicadas na pesquisa: 
%$( \mu,\lambda )$ ou $( \mu + \lambda )$ e Geracional ou Estado Estável 
%(em inglês, \textit{steady state}), respectivamente.

\begin{table}[tp]
	\centering
	\caption{Termos comuns na Computação Evolucionária}
	\label{tbl:evo_dict}
	\begin{tabularx}{\linewidth}{|X|X|}
		\hline
		\textbf{Termo} & \textbf{Significado} \\
		\hline
		Indivíduo & Uma solução candidata \\
		\hline
		Filha e Pai & Uma solução Filha é uma cópia alterada de uma solução Pai \\
		\hline
		População & Um conjunto de soluções candidatas \\
		\hline
		Aptidão & Qualidade de um indivíduo (Solução) \\
		\hline
		Seleção & Coletar indivíduos baseados na sua aptidão \\
		\hline
		Mutação & Realização de pequenas alterações em indivíduos. Também chamada de reprodução assexuada \\
		\hline
		Recombinação (em inglês, \textit{crossover}) & Uma forma de alteração especial que recebe como parâmetro duas soluções pais e (normalmente) produz duas soluções filhas \\
		\hline
		Reprodução & O ato de produzir uma ou mais soluções filhas a partir de uma solução pai \\
		\hline
		Geração & Um ciclo de medida de aptidão ou de reprodução de uma população \\
		\hline
%		\citep{Machado:2002:MPE:1765317.1765332} & Patrulha Multiagente "Padrão" & Simulação & Ociosidade instantânea do nó, ociosidade instantânea do grafo, ociosidade do grafo, ociosidade máxima e tempo de exploração \\
%		\hline
	\end{tabularx}
	\caption*{Fonte: Adaptado de \citep{Luke2013Metaheuristics}}
\end{table}

\section{Estratégias Evolucionárias}

As duas estratégias evolucionárias usadas no presente trabalho diferem entre si 
pela forma como fazem a composição entre a população de pais e a população de 
filhos para construir a nova geração, que será utilizada na iteração seguinte do 
algoritmo.

A primeira \ac{es} é conhecida como $( \mu, \lambda )$. Tipicamente, começa-se 
com uma população de $\lambda$ indivíduos gerados de forma arbitrária. Nessa 
\ac{es}, o $\mu$ representa o número de pais cujos filhos serão usados para 
compor a próxima população que também deve ter $\lambda$ indivíduos no total. 
Note que $\lambda$ tem que ser um múltiplo de $\mu$. Então, os $\mu$ indivíduos 
mais aptos são escolhidos, processo chamado de Seleção por Truncamento (em 
inglês, \textit{Truncate Selection}). Os indivíduos selecionados sofrem mutação 
para gerar $ \lambda / \mu $ filhos cada . O que acarretará em uma nova 
população de $\lambda$ indivíduos que será a geração utilizada na próxima 
iteração do algoritmo \citep{Luke2013Metaheuristics}.

O Pseudocódigo~\ref{mu_lambda_es} exemplifica a Estratégia Evolucionária 
$( \mu, \lambda )$.

\begin{algorithm}                      % enter the algorithm environment
	\caption{Estratégia Evolucionária $( \mu, \lambda )$}          % give the algorithm a caption
	\label{mu_lambda_es}                           % and a label for \ref{} commands later in the document
	\algcomment{\begin{center} Fonte: Adaptado de \citep{Luke2013Metaheuristics} \end{center}}
	\begin{algorithmic}                    % enter the algorithmic environment
		\Procedure{$ \mu\_\lambda $\_ES}{$ \mu, \lambda $}
		\State $P \gets \{\} $
		\For{$1 ... \lambda$}
			\State $p \gets$ \{novo indivíduo gerado de forma arbitrária\}
			\State CalculaAptidão($p$)
			\State $P \gets P \ \cup $ \{$p$\} 
		\EndFor
		\State $Melhor \gets$ nulo
		\Repeat
			\State Ordena($P$) \Comment{Ordena a população de acordo com a aptidão}
			\State $Q \gets P_{1...\mu}$ \Comment{Inicia $Q$ com os $\mu$ indivíduos mais aptos}
			\State $P \gets \{\}$
			\For{$Q_{j} \in Q$}
				\For{$1... \lambda / \mu$}
					\State $P \gets P \ \cup $ \{Aplica-Mutação($Q_{j}$)\}
				\EndFor
			\EndFor
			\For{$P_{i} \in P$}
				\State CalculaAptidão($P_{i}$)
				\If{$Melhor = $ nulo \textbf{ou} Aptidão($P_{i}$) > Aptidão($Melhor$)}
					\State $Melhor \gets P_{i}$
				\EndIf
			\EndFor
		\Until{não tenhamos mais tempo}
		\State \textbf{Retorne} $Melhor$
		\EndProcedure
	\end{algorithmic}
\end{algorithm}

A segunda \ac{es} utilizada nesta pesquisa é chamada de $( \mu + \lambda)$. 
Enquanto que na estratégia $( \mu, \lambda )$ cada pai é substituído pelos seus 
$ \lambda / \mu $ filhos, na estratégia $( \mu + \lambda )$, os $\mu$ pais 
permanecem para competir com seus $ \lambda$ filhos na iteração seguinte. 
\citep{Luke2013Metaheuristics} aponta que isso geralmente faz com que a 
estratégia $( \mu + \lambda )$ explore mais os ótimos locais em comparação com 
a estratégia $( \mu, \lambda )$, já que um pai suficientemente apto pode fazer 
com que a \ac{es} fique "presa"\ em seus descendentes imediatos, causando uma 
convergência das populações para o ótimo local ao redor do pai.

O Pseudocódigo~\ref{mu+lambda_es} ilustra a Estratégia Evolucionária 
$( \mu + \lambda )$.

\begin{algorithm}                      % enter the algorithm environment
	\caption{Estratégia Evolucionária $( \mu + \lambda )$}          % give the algorithm a caption
	\label{mu+lambda_es}                           % and a label for \ref{} commands later in the document
	\algcomment{\begin{center} Fonte: Adaptado de \citep{Luke2013Metaheuristics} \end{center}}
	\begin{algorithmic}                    % enter the algorithmic environment
		\Procedure{$ \mu+\lambda $\_ES}{$ \mu, \lambda $}
		\State $P \gets \{\} $
		\For{$1 ... \lambda$}
			\State $p \gets$ \{novo indivíduo gerado de forma arbitrária\}
			\State CalculaAptidão($p$)
			\State $P \gets P \ \cup $ \{$p$\} 
		\EndFor
			\State $Melhor \gets$ nulo
		\Repeat
			\State Ordena($P$) \Comment{Ordena a população de acordo com a aptidão}
			\State $Q \gets P_{1...\mu}$ \Comment{Inicia $Q$ com os $\mu$ indivíduos mais aptos}
			\State $P \gets Q$ \Comment{A diferença está aqui}
			\For{$Q_{j} \in Q$}
				\For{$1... \lambda / \mu$}
					\State $P \gets P \ \cup $ \{Aplica-Mutação($Q_{j}$)\}
				\EndFor
			\EndFor
			\For{$P_{i} \in P$}
				\State CalculaAptidão($P_{i}$)
				\If{$Melhor = $ nulo \textbf{ou} Aptidão($P_{i}$) > Aptidão($Melhor$)}
					\State $Melhor \gets P_{i}$
				\EndIf
			\EndFor
		\Until{não tenhamos mais tempo}
		\State \textbf{Retorne} $Melhor$
		\EndProcedure
	\end{algorithmic}
\end{algorithm}

É importante notar que as Estratégias Evolucionárias não utilizam a recombinação 
para gerar novos indivíduos. Para isso, elas utilizam apenas a Mutação cuja 
implementação depende do tipo de dados envolvido no problema que está sendo 
estudado. No próximo capítulo, serão propostos alguns operadores de Mutação para 
a \ac{tmap}, que irão agir sobre caminhos e ciclos de um grafo. Também será 
tratado, no próximo capítulo, como foram implementadas as funções de gerar 
indivíduos e calcular suas aptidões dentro do contexto da \ac{tmap}.

\section{Algoritmos Genéticos}

Os Algoritmos Genéticos são similares às \acp{es}: o seu \textit{loop} principal 
consiste em selecionar indivíduos da polução de acordo com as respectivas 
aptidões, reproduzir os indivíduos selecionados e iterar sobre a nova população 
para calcular as novas aptidões \citep{Luke2013Metaheuristics}.

No entanto, eles são diferentes na forma como selecionam e reproduzem os 
indivíduos. Enquanto que nas Estratégias Evolucionárias, todos os pais são 
selecionados simultaneamente e passam por mutação em seguida, os Algoritmos 
Genéticos selecionam os pais e geram filhos ao poucos até que se tenha uma 
população de filhos suficiente. A reprodução em si também é bem diferente das 
\acp{es}: nos Algoritmos Genéticos, inicia-se com uma população de filhos vazia. 
O Algoritmo então seleciona dois pais de forma arbitrária, recombina esses pais 
em dois novos indivíduos filhos e então faz a mutação deles (as \acp{es} só 
realizam a mutação). Esses dois novos indivíduos são adicionados à população. 
Esse processo se repete até que a população de filhos esteja completamente 
preenchida \citep{Luke2013Metaheuristics}.

O pseudocódigo~\ref{GenAlg} deve ajudar o leitor a compreender o algoritmo.

\begin{algorithm}[h]                      % enter the algorithm environment
	\caption{Algoritmo Genético}          % give the algorithm a caption
	\label{GenAlg}                           % and a label for \ref{} commands later in the document
	\algcomment{\begin{center} Fonte: Adaptado de \citep{Luke2013Metaheuristics} \end{center}}
	\begin{algorithmic}                    % enter the algorithmic environment
		\Procedure{AlgoritmoGenetico}{$tamanhoPopulação$}
		\State $P \gets \{\} $
		\For{$1 ...\ tamanhoPopulação$}
			\State $p \gets$ \{novo indivíduo gerado de forma arbitrária\}
			\State CalculaAptidão($p$)
			\State $P \gets P \ \cup $ \{$p$\} 
		\EndFor
		\State $Melhor \gets$ nulo
		\Repeat
			\State $Q \gets \{\}$
			\For{$1 ...\ tamanhoPopulação / 2$}
				\State $Pai_{a} \gets $ Seleciona($P$)
				\State $Pai_{b} \gets $ Seleciona($P$)
				\State $Filho_{a},\ Filho_{b} \gets $ Recombina($Pai_{a}$, $Pai_{b}$)
				\State $Q \gets Q\ \cup $ \{Aplica-Mutação($Filho_{a}$), Aplica-Mutação($Filho_{b}$)\}
			\EndFor
			\State $P \gets Q$
			\For{$P_{i} \in P$}
				\State CalculaAptidão($P_{i}$)
				\If{$Melhor = $ nulo \textbf{ou} Aptidão($P_{i}$) > Aptidão($Melhor$)}
					\State $Melhor \gets P_{i}$
				\EndIf
			\EndFor
		\Until{não tenhamos mais tempo}
		\State \textbf{Retorne} $Melhor$
		\EndProcedure
	\end{algorithmic}
\end{algorithm}

Esta é a forma tradicional como o Algoritmo Genético é apresentado nos livros 
textos \citep{Luke2013Metaheuristics}, também conhecido como Algoritmo Genético 
Geracional. No entanto, existem outras variações, como por exemplo, o Algoritmo 
Genético de Estado Estável, do inglês, \textit{Steady State Genetic Algorithm}.
O \ac{ga} Geracional é assim conhecido pois nele a população é completamente 
atualizada de uma vez. O \ac{ga} de Estado Estável, por sua vez, introduz um 
ou dois filhos, obtidos através de recombinação e mutação, diretamente na 
população de indivíduos (eliminando outros indivíduos para abrir espaço) e segue 
para a próxima geração.

Dessa forma, assim como a \ac{es} $( \mu + \lambda )$, os indivíduos pais se 
mantém na população e disputam com os filhos nas gerações seguintes. Por isso, 
este algoritmo pode sofrer do mesmo problema e ficar "preso"\ a ótimos locais ao 
redor dos pais \citep{Luke2013Metaheuristics}.

O pseudocódigo~\ref{GenAlgSteady} detalha as características do Algoritmo 
Genético de Estado Estável.

\begin{algorithm}[h]                      % enter the algorithm environment
	\caption{Algoritmo Genético de Estado Estável}          % give the algorithm a caption
	\label{GenAlgSteady}                           % and a label for \ref{} commands later in the document
	\algcomment{\begin{center} Fonte: Adaptado de \citep{Luke2013Metaheuristics} \end{center}}
	\begin{algorithmic}                   % enter the algorithmic environment
		\Procedure{AlgoritmoGeneticoEstadoEstavel}{$tamanhoPopulação$}
		\State $P \gets \{\} $
		\For{$1 ...\ tamanhoPopulação$}
			\State $P \gets P \ \cup $ \{novo indivíduo gerado de forma arbitrária\} 
		\EndFor
		\For{$P_{i} \in P$}
			\State CalculaAptidão($P_{i}$)
			\If{$Melhor = $ nulo \textbf{ou} Aptidão($P_{i}$) > Aptidão($Melhor$)}
				\State $Melhor \gets P_{i}$
			\EndIf
		\EndFor
		\Repeat
			\State $Pai_{a} \gets $ Seleciona($P$)
			\State $Pai_{b} \gets $ Seleciona($P$)
			\State $Filho_{a},\ Filho_{b} \gets $ Recombina($Pai_{a}$, $Pai_{b}$)
			\State $Filho_{a} \gets $ Aplica-Mutação($Filho_{a}$)
			\State $Filho_{b} \gets $ Aplica-Mutação($Filho_{b}$)
			\State CalculaAptidão($Filho_{a}$)
			\If{Aptidão($Filho_{a}$) > Aptidão($Melhor$)}
				\State $Melhor \gets Filho_{a}$
			\EndIf
			\State CalculaAptidão($Filho_{b}$)
			\If{Aptidão($Filho_{b}$) > Aptidão($Melhor$)}
				\State $Melhor \gets Filho_{b}$
			\EndIf
			\State $p_{c} \gets $ SelecionaParaMorrer($P$) \Comment{Seleciona um indivíduo para remover da população}
			\State $p_{d} \gets $ SelecionaParaMorrer($P$) \Comment{$p_{c} \neq p_{d}$}
			\State $P \gets P\ -\ \{p_{c},p_{d}\}$
			\State $P \gets P\ \cup\ \{Filho_{a},\ Filho_{b}\}$
		\Until{não tenhamos mais tempo}
		\State \textbf{Retorne} $Melhor$
		\EndProcedure
	\end{algorithmic}
\end{algorithm}

O algoritmo genético de estado estável utilizado na presente pesquisa utiliza 
o operador de seleção chamado Seleção da Pior Aptidão 
\citep{Luke2013Metaheuristics} para escolher os indivíduos da antiga geração 
que darão lugar aos novos. Esse operador simplesmente escolhe o indivíduo que 
possuir a menor aptidão dentro da população.

No próximo capítulo, serão explorados as aplicações dos algoritmos acima 
apresentados para a \ac{tmap}. Serão propostos operadores de mutação e de 
recombinação que possam ser aplicados ao modelo utilizado, além de um método 
para calcular a aptidão de um indivíduo. No entanto, pode-se, ainda neste 
capítulo, tratar de um operador utilizado nos algoritmos genéticos que é 
definido sem interferência do problema ou da estrutura de dados no qual ele 
está modelado. Os operadores de Seleção podem ser descritos para uma população 
qualquer de indivíduos modelados de forma arbitrária.

Na presente pesquisa, foi utilizado o operador de seleção comumente chamado de 
Torneio (do inglês, \textit{Tournament}) \citep{Luke2013Metaheuristics}. Ele 
consiste em coletar, da população, $t$ indivíduos de forma aleatória. Destes, 
o algoritmo retorna apenas o indivíduo mais apto. Segue o pseudocódigo para 
ilustrar esse operador.

\begin{algorithm}[h]                      % enter the algorithm environment
	\caption{Seleção por Torneio}          % give the algorithm a caption
	\label{torneio}                           % and a label for \ref{} commands later in the document
	\algcomment{\begin{center} Fonte: Adaptado de \citep{Luke2013Metaheuristics} \end{center}}
	\begin{algorithmic}                    % enter the algorithmic environment
		\Procedure{Torneio}{$P, t$}
		\State $Melhor \gets $ indivíduo escolhido aleatoriamente de $P$ \Comment{De forma que não possa ser escolhido uma segunda vez}
		\For{$i$ de $2 ... t$}
			\State $Proximo \gets $ indivíduo escolhido aleatoriamente de $P$
			\If{Aptidão($Proximo$) $>$ Aptidão($Melhor$)}
				\State $Melhor \gets Proximo$
			\EndIf
		\EndFor
		\State \textbf{Retorne} $Melhor$
		\EndProcedure
	\end{algorithmic}
\end{algorithm}

Neste capítulo foram revisados quatro algoritmos evolucionários presentes na 
literatura que foram utilizados nesta pesquisa. Todos esses algoritmos são bases 
para resolver problemas de otimização discretos ou contínuos. No entanto, eles 
precisam ser complementados com operadores para: criação de indivíduos, mutação 
e recombinação. Estes operadores dependem do problema, influenciam diretamente 
na busca realizada por esses algoritmos e podem ser um fator determinante para o 
sucesso ou fracasso da abordagem evolucionária \citep{Luke2013Metaheuristics}.

Dessa forma, no capítulo seguinte, serão apresentados alguns operadores que 
podem ser utilizados para mutação, recombinação e criação de indivíduos no 
contexto da \ac{tmap}.